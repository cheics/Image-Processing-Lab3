\documentclass[article, 1.5space, letterpaper, 12pt, oneside, header, footer]{SydeClass}
\graphicspath{{images/}}
\usepackage{subfigure}
\usepackage{eqnarray}


% --------- Title Info -----------
\titlestyle{design} % used in SydeTitle.tex. Can equal one of the following values: design, work

\title{Lab 1}
\subtitle{Fundamentals of Image Processing}

\coursecode{SYDE 475}
\department{Systems Design Engineering}

\author{Colin Heics, 20240543}
\authorheader{C. Heics}
\authortwo{Neil Sokol, 20265064}
\authorheadertwo{N. Sokol}

\date{\today}
\instructor{Alex Wong}

\subsectionfont{\normalsize}
\setcounter{secnumdepth}{2}
\setcounter{tocdepth}{1}

\input{matlabFormating}

% ############  ############
\begin{document}

% ---------- Title ------------

%% Use the command "
%% Use the command "
%% Use the command "\input{SydeTitle}" in your main file to include this file.

\begin{titlepage}
	\makeatletter % use .cls usage for <at>
	
	\pagestyle{empty}
	\equalmargins
	
	\ifthenelse{\equal{\@titlestyle}{work}}{
		\begin{center}
			\vspace*{2em}

			University of Waterloo\\
			Faculty of Engineering\\
			Department of Systems Design Engineering

			\null\vfill
		
			\Huge\@title \\
			\ifdefined \@subtitle \Large\@subtitle \\ \fi
			\normalsize

			\null\vfill
		
			\@company\\
			\@companyaddress \vspace{2em}
		
			\@author\\
			\@date
		\end{center}
	}{\relax} % end if
	
	\ifthenelse{\equal{\@titlestyle}{design}}{
		\begin{center}
			\vspace*{5em}
	
			\Huge\@title \\
			\ifdefined \@subtitle \Large\@subtitle \\ \fi
			\normalsize
	
			\vfill
		
			A Report Submitted in Partial Fulfilment\\
			of the Requirements for \@coursecode \vspace{4em}
		
			\ifdefined \@groupname \@groupname \\ \fi
		  \@author \\
			\ifdefined \@authortwo \@authortwo \\ \fi
			\ifdefined \@authorthree \@authorthree \\ \fi
			\ifdefined \@authorfour \@authorfour \\ \fi
		  \vspace{3em}
		
			Faculty of Engineering \\
			\ifdefined \@department Department of \@department \\ \fi
			\vspace{3em}
		
			\@date \\
			
			\ifdefined \@instructor Course Instructor: \@instructor \\ \fi
			\ifdefined \@supervisor Project Supervisor: \@supervisor \\ \fi
			
		\end{center}
	}{\relax} % end if
	
	\makeatother % return to document usage for <at>
\end{titlepage}

%\pagestyle{plain}
%\offsetmargins" in your main file to include this file.

\begin{titlepage}
	\makeatletter % use .cls usage for <at>
	
	\pagestyle{empty}
	\equalmargins
	
	\ifthenelse{\equal{\@titlestyle}{work}}{
		\begin{center}
			\vspace*{2em}

			University of Waterloo\\
			Faculty of Engineering\\
			Department of Systems Design Engineering

			\null\vfill
		
			\Huge\@title \\
			\ifdefined \@subtitle \Large\@subtitle \\ \fi
			\normalsize

			\null\vfill
		
			\@company\\
			\@companyaddress \vspace{2em}
		
			\@author\\
			\@date
		\end{center}
	}{\relax} % end if
	
	\ifthenelse{\equal{\@titlestyle}{design}}{
		\begin{center}
			\vspace*{5em}
	
			\Huge\@title \\
			\ifdefined \@subtitle \Large\@subtitle \\ \fi
			\normalsize
	
			\vfill
		
			A Report Submitted in Partial Fulfilment\\
			of the Requirements for \@coursecode \vspace{4em}
		
			\ifdefined \@groupname \@groupname \\ \fi
		  \@author \\
			\ifdefined \@authortwo \@authortwo \\ \fi
			\ifdefined \@authorthree \@authorthree \\ \fi
			\ifdefined \@authorfour \@authorfour \\ \fi
		  \vspace{3em}
		
			Faculty of Engineering \\
			\ifdefined \@department Department of \@department \\ \fi
			\vspace{3em}
		
			\@date \\
			
			\ifdefined \@instructor Course Instructor: \@instructor \\ \fi
			\ifdefined \@supervisor Project Supervisor: \@supervisor \\ \fi
			
		\end{center}
	}{\relax} % end if
	
	\makeatother % return to document usage for <at>
\end{titlepage}

%\pagestyle{plain}
%\offsetmargins" in your main file to include this file.

\begin{titlepage}
	\makeatletter % use .cls usage for <at>
	
	\pagestyle{empty}
	\equalmargins
	
	\ifthenelse{\equal{\@titlestyle}{work}}{
		\begin{center}
			\vspace*{2em}

			University of Waterloo\\
			Faculty of Engineering\\
			Department of Systems Design Engineering

			\null\vfill
		
			\Huge\@title \\
			\ifdefined \@subtitle \Large\@subtitle \\ \fi
			\normalsize

			\null\vfill
		
			\@company\\
			\@companyaddress \vspace{2em}
		
			\@author\\
			\@date
		\end{center}
	}{\relax} % end if
	
	\ifthenelse{\equal{\@titlestyle}{design}}{
		\begin{center}
			\vspace*{5em}
	
			\Huge\@title \\
			\ifdefined \@subtitle \Large\@subtitle \\ \fi
			\normalsize
	
			\vfill
		
			A Report Submitted in Partial Fulfilment\\
			of the Requirements for \@coursecode \vspace{4em}
		
			\ifdefined \@groupname \@groupname \\ \fi
		  \@author \\
			\ifdefined \@authortwo \@authortwo \\ \fi
			\ifdefined \@authorthree \@authorthree \\ \fi
			\ifdefined \@authorfour \@authorfour \\ \fi
		  \vspace{3em}
		
			Faculty of Engineering \\
			\ifdefined \@department Department of \@department \\ \fi
			\vspace{3em}
		
			\@date \\
			
			\ifdefined \@instructor Course Instructor: \@instructor \\ \fi
			\ifdefined \@supervisor Project Supervisor: \@supervisor \\ \fi
			
		\end{center}
	}{\relax} % end if
	
	\makeatother % return to document usage for <at>
\end{titlepage}

%\pagestyle{plain}
%\offsetmargins

% ############ Chapters ############
\pagenumbering{arabic}

\section{Noise Generation}

When evaluating image processing algorithms, it is important to see how they perform under various levels of noise.


\begin{figure}[ht]
\centering
	\subfigure[Original image]{
	\includegraphics[width=0.45\linewidth]{question2/toy}
	}
\end{figure}

\begin{figure}[ht]
\centering
	\subfigure[Image with Gaussian Noise]{
	\includegraphics[width=0.45\linewidth]{question2/gauss}
	}
	\subfigure[Histogram of Gaussian Noise Image]{
	\includegraphics[width=0.45\linewidth]{question2/gauss_hist}
	}
	\subfigure[Image with Speckle Noise]{
	\includegraphics[width=0.45\linewidth]{question2/speckle}
	}
	\subfigure[Histogram of Speckle Noise Image]{
	\includegraphics[width=0.45\linewidth]{question2/speckle_hist}
	}
	\caption{Toy image with Gaussian and Speckle Noise}
	\label{fig:noiseGeneration.toy}
\end{figure}
 

\subsection{Discussion questions}

\subsubsection{ Describe each of the histograms in the context of the corresponding noise models. Why do they appear
that way?}
In the gaussian histogram, image intensities are grouped in gaussian distributions centered around the two intesity values of the original image. It appears the way it is because as additive noise, all that is being done is the gaussian distribution is added to the image. Speckled noise, which is multiplicative, appears to be a flat distribution centered around the original intensities. There is more variance around the higher original intesity value, which is due to the multiplicative nature of speckle noise.

\subsubsection{ Are there visual differences between the noise contaminated images? What are they? Why}



\appendix
\newpage


\section{Noise Reduction in the Spatial Domain}

\clearpage
\subsection{Section0}
\begin{figure}[ht]
\centering
	\subfigure[Original image]{
	\includegraphics[width=0.45\linewidth]{question3/0_lenaBase}
	}
	\subfigure[Histogram]{
	\includegraphics[width=0.45\linewidth]{question3/0_lenaBase_hist}
	}
	\subfigure[Image with gaussian noise $\mu$=0, $\sigma$=0.002; PSNR +26.99dB]{
	\includegraphics[width=0.45\linewidth]{question3/0_lenaNoisyGauss}
	}
	\subfigure[Histogram]{
	\includegraphics[width=0.45\linewidth]{question3/0_lenaNoisyGauss_hist}
	}
	\caption{Images with gaussian noise}
	\label{fig:gaussianNoise}
\end{figure}


\clearpage
\subsection{Section1}
\begin{figure}[ht]
\centering
	\subfigure[Denoised image with 3x3 average kernel; PSNR +18.42dB]{
	\includegraphics[width=0.45\linewidth]{question3/1_lenaDeNoisyGauss3x3avg}
	}
	\subfigure[Histogram]{
	\includegraphics[width=0.45\linewidth]{question3/1_lenaDeNoisyGauss3x3avg_hist}
	}
	
	\caption{Images with gaussian noise}
	\label{fig:gaussianNoise}
\end{figure}

\clearpage
\subsection{Section2}
\begin{figure}[ht]
\centering
	\subfigure[Denoised image with 7x7 average kernel; PSNR +25.52dB]{
	\includegraphics[width=0.45\linewidth]{question3/2_lenaDeNoisyGauss7x7avg}
	}
	\subfigure[Histogram]{
	\includegraphics[width=0.45\linewidth]{question3/2_lenaDeNoisyGauss7x7avg_hist}
	}
	
	\caption{Images with gaussian noise}
	\label{fig:gaussianNoise}
\end{figure}

\clearpage
\subsection{Section3}
\begin{figure}[ht]
\centering
	\subfigure[Denoised image with 7x7 gaussian kernel; PSNR +27.08dB]{
	\includegraphics[width=0.45\linewidth]{question3/3_lenaDeNoisyGauss7x7avgGauss}
	}
	\subfigure[Histogram]{
	\includegraphics[width=0.45\linewidth]{question3/3_lenaDeNoisyGauss7x7avgGauss_hist}
	}
	
	\caption{Images with gaussian noise}
	\label{fig:gaussianNoise}
\end{figure}


\clearpage
\subsection{Section4}

\begin{figure}[ht]
\centering
	\subfigure[S\&P noise on original image; PSNR +18.41dB]{
	\includegraphics[width=0.45\linewidth]{question3/4_lenaNoisySp}
	}
	\subfigure[Histogram]{
	\includegraphics[width=0.45\linewidth]{question3/4_lenaNoisySp_hist}
	}
\end{figure}

\begin{figure}[ht]
\centering	
	\subfigure[S\&P denoised with 7x7 average kernel; PSNR +25.52dB]{
	\includegraphics[width=0.45\linewidth]{question3/4_lenaDeNoisySp_7x7Avg}
	}
	\subfigure[Histogram]{
	\includegraphics[width=0.45\linewidth]{question3/4_lenaDeNoisySp_7x7Avg_hist}
	}
\end{figure}

\begin{figure}[ht]
\centering
	\subfigure[S\&P denoised with 7x7 gaussian kernel; PSNR +27.08dB]{
	\includegraphics[width=0.45\linewidth]{question3/4_lenaDeNoisySp_7x7AvgGauss}
	}
	\subfigure[Histogram]{
	\includegraphics[width=0.45\linewidth]{question3/4_lenaDeNoisySp_7x7AvgGauss_hist}
	}
\end{figure}


\clearpage
\subsection{Section5}

\begin{figure}[ht]
\centering
	\subfigure[S\&P denoised with median filter; PSNR +34.47]{
	\includegraphics[width=0.45\linewidth]{question3/5_lenaDeNoisyMedian}
	}
	\subfigure[Histogram]{
	\includegraphics[width=0.45\linewidth]{question3/5_lenaDeNoisyMedian_hist}
	}
\end{figure}

% -------- Bibliography --------
%\addcontentsline{toc}{chapter}{\hspace{13pt} References}
\bibliography{refs}

\end{document}  